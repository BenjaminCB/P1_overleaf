\pdfbookmark[0]{Title page}{label:titlepage}
\aautitlepage{%
  \englishprojectinfo{
    \rtitle %title
  }{%
    \rtheme %theme
  }{%
    \rperiod %project period
  }{%
    \rgroupnum % project group
  }{
    \GetAuthorList[L]{}
  }{
    \rsupervisor
  }{%
    1 % number of printed copies
  }{%
    \today % date of completion
  }%
}{%department and address
  \textbf{\rstudy}\\
  Aalborg University\\
  \href{http://www.aau.dk}{http://www.aau.dk}
}{% the abstract
 The goal of this project was to create a program which could automate the process of generating work schedules at \siemens. To do that, it has been necessary to analyse \siemens as an organisation and the needs of the employees. Furthermore, it was also important to understand the local union agreements, so the schedules could be made accordingly. Fairness was also an important factor, since it could make the difference between a functional schedule and a great schedule. The analysis of these areas was essential to the solution development. In the second section of the report, we explain the structure of our program both overall, as well as in detail. We also present a thorough and detailed documentation of the implementation and tests of our program. Lastly, we conclude that we have made a program that is far more efficient than \siemens' current method of work scheduling.
}