\section{Existing solutions}
In this section we will be looking at pre-existing solutions to see how these solutions approach the problem. By doing this we will be able to find out which things we should include, which things we should try to improve upon, and what we should avoid doing.

We will in addition to the above be examining the solution currently in use at \siemens, for the sake having a reference to compare other existing solutions as well as the solution we will be developing later in the project.

\subsection{Current solution at \siemens}
This is the method that is currently being used in our case study. Once a year, one person, who is responsible for time planning, sits down and makes the plan for the entire next year. Making the schedule takes about one day for an experienced worker, however the plan has to be accepted by the employers and the employees for it to take effect. If it does not get accepted, that means there needs to be made a new one, with a new system, so that it differs from the not-accepted previous schedule. A program would also be useful later, when a new worker might have to make schedules, after the old worker resigned, saving even more time.

%Not only does this take an excessive amount of time, which could be saved if a program was to be made instead, but it is also some of the most mundane and repetitive work there is to do. 


\subsubsection{Microsoft Excel}
Currently, Microsoft Excel is the preferred method of time-planning for \siemens, this is because of its cell and row system, which makes it easy to understand at a quick glance, but it also uses a mathematical function system, which allows calculations which can be dynamic, meaning that if you change a number in one cell, linked cells can get changed as well.

Currently the time-planner is given a schedule, following the example in Table \ref{tab:exampleSchedule}, which can be seen on page \pageref{tab:exampleSchedule}. But this schedule does not account for vacation, public holidays like Easter or the number of hours per year the teams should work. Therefore, these things need to be fixed, and time slots in the schedule need to be either removed, or added, to make sure production keeps happening, and that the union agreement is upheld and the workers are happy.

\subsection{TimePlan}
TimePlan is a piece of software developed by TimePlan Software A / S. The program can be used to create work schedules either manually or automatically. In addition to this the program can also warn the user in case the work schedule is in breach of either a union agreement or other relevant laws. The system also supports tracking of overtime, absence, vacation, holidays, and more. All these various pieces of information can then be exported to be passed to other separate systems for payroll processing. \parencite{TimePlan}

\subsection{Planday}
Planday is a piece of scheduling-software developed by the company of the same name. It is used by several major companies such as Best Western, Just Eat, and Wolford. Planday helps the person(s) in charge of creating the schedule by making factors such as vacation, availability, etc. more accessible and easier to organize. It is therefore not completely automated, since the organizer must still finish the schedule by himself/herself, but Planday's tools makes this process much easier and less time consuming. Planday also makes it possible for employees to make suggestions themselves. \parencite{planday}

\subsection{Smartplan}
Smartplan is a system that aids in planning and managing a work schedule. Like Planday, Smartplan is not automated so the end user will still be doing some of the work. According to their website, it does offer various functions that attempt to make the process easier. It allows creating schedule templates that can be copied and filled in so that the general framework for the schedule can be reused.
It can track the number of hours and other statistics that can used to set up rules in the system that it will then enforce for the schedule. It allows the employees to make request in addition to allowing them to take up open shifts in the schedule. While the system is not fully automatic it is capable of making suggestions based on the given rules and employee requests.

Much like TimePlan, Smartplan also offers functionality for exporting payroll information and is capable of tracking vacation, absence, and leave. Smartplan also offers functionality that lies slightly outside the scope of work schedule planning such as a clock-in (time tracking) system and general-purpose internal communications system. \parencite{smartplan}

\subsection{Other options}
Other software or websites for scheduling are mainly used for smaller groups, retail, and restaurants. Where the product is service, this means the scheduling is largely based on when a given customer base is normally there e.g 18:00-20:00 at a restaurant where more waiters and such are needed. The software is made to accommodate this, and schedules are made within given groups for example, we need six waiters between 16:00 and 22:00 Monday till Thursday and 10 waiters from Friday till Sunday, the software would then use this information to make a schedule for everyone and give them the number of hours specified, this kind of schedules are typically made for a couple of weeks at a time and switched between people to keep the scheduling "fair".

% summary and conclusion (what do the solutions have in common, how do they differ, what have we learned from it)
\subsection{Summary}
A general trend that seems to appear between many of the pre-existing solutions is that there are a lot of interconnecting systems besides just work schedule planning. This makes sense since something like a payroll is intrinsically tied to the amount of regular and overtime hours an employee has worked along with knowing whether the employee has worked during a time slot, where they are entitled to additional pay. Furthermore, these programs still require some time spent on setting up everything, because they are not tailored directly to the shifting teams of \siemens. Therefore our solution will be more suitable, as it will output a schedule that \siemens already use, while not requireing any setup time each year. Our system, however will not be able to handle all the extra tasks like pay.