\section{Fairness}
In this section we will give a basic introduction to some thought processes that are fair, giving anyone reading this a base line understanding of what fairness is. We will also describe which of these thought processes we want to implement and the issues that could arise with doing so.  

One thing we do not want is an unfair work schedule, though it will most likely be impossible to make everyone completely satisfied with their schedule. We want everyone to feel as though they have not been treated differently than anyone else. To make sure that everyone has a good starting point as to what fairness is, here is the definition from Merriam-Webster: 
\begin{quote}
    especially : fair or impartial treatment : lack of favouritism toward one side or another
    \parencite{merriam-webster_definition_nodate}
\end{quote}
This is along the lines of what was just mentioned, but there are a lot of ways to make a work schedule without picking favourites. One way to do it is by just giving people random days. In fact, this might be one of the fairest ways to do it because there is no chance of the employer picking favourites even subconsciously. While this might be fair, it is not necessarily satisfying for the employees, none of their wishes have been considered, and there is no guarantee that any of them will come to fruition. Therefore, we need to look at some different ways to be fair and figure out what will work best in our situation. In a short article by Arthur Dobrin \parencite{dobrin_its_2012} he presents three ways of being fair that we probably all have thought about or seen in our own lives. 
\begin{enumerate}
    \item Sameness: Everyone gets the same
    \item Deservedness: You get what you put in
    \item Need: You get what you need
\end{enumerate}
An example for any of these could be tax and public welfare. In a situation where sameness was the priority everyone would pay an equal amount, and everyone would get an equal amount back. Someone with a high salary and someone with minimum wage would pay an equal amount, a world-class athlete would get the same help from the public welfare as someone with a handicap.
If we instead went for deservedness, we probably would not pay taxes because you have earned that money through hard work and effort, therefore you deserve it. 
Lastly when everyone gets what they needed you can get the help in difficult situations and the people with the most money pay more in tax for the common good.
Often what we see is that there is a mix of these different thought processes, and we will probably have to do that as well. 
Something that we would like is the ability to take worker wishes into account. However, this might not be an excellent idea to implement from the beginning since workers at \siemens can wish for time off with a week ahead of time, and schedule is created a year ahead of time. So, it is hard to say how many workers would wish for a day of, so far ahead of time, it would probably only be used for special occasions like weddings and such. Therefore, it could end up being a feature that we spend a lot of time on, but still end up being one of the least useful. However, we definitely have this as one of the things to implement if we end up having extra time. 

We still have some other special days that need to be sorted, that being public holidays and Saturdays. For the most part \siemens does not actually want people to work public holidays since they have to pay extra in those cases, but workers also have to work a specific amount of time, so sometimes it is necessary. Therefore, work on public holidays should be distributed fairly. Another thing to note about public holidays is that if a worker is set to be working too much in the proposed work schedule, giving them time off on a public holiday should be prioritized, because it would be more cost effective. 

The other day is Saturday because as this is one of the common for everyone else to have time of as well. As you will see in section about interested parties spending time with family and friends is very important for mental health and well-being. It is not a guarantee that shifting team workers can spend time with family or friends on a regular basis, therefore it is important that we distribute time off on Saturdays fairly. 

If we look at the mentioned thought processes from before, we can do some simple analysis and get an idea of how we want to approach this. 
We cannot really do it with deservedness too easily, since we cannot really give preferential treatment based on their jobs, since it should be the case that every job in the workplace has equal value.
Some things could be done based need. In the instance where a worker is working too much, they would have a need to get time off, and as mentioned earlier in a situation like that, public holidays should be prioritized. Other than that, though workers mostly have the same amount of need to get a day off. 

The only thing left is to use the sameness type of fairness, meaning everyone would get the same number of good days off. This could work, but it seems very inefficient. Luckily, we are dealing with adults, that should understand that making a work schedule that is 100\% fair in every area would be a little unreasonable. So, what we probably want is not the fairest schedule or the most efficient, but one where we try to maximize the number of good workdays for each worker, while keeping the difference of each worker to a minimum. That way most people would be in the same position, and the ones that are not would not be far from everyone else and it would not be based on anything. 
%lige antal loerdage
%ikke nogle wishes
%lige antal helligdage
%fjern vagter fra helligdage hvis der er for mange vagter

%how sholud we distribute the days off. Is it fair is you get all your days off in a short period of time. 