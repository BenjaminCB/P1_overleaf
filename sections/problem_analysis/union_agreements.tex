\section{Union Agreements}

A very important factor to take into account when discussing the employees' rights and their shifts, are the unions. The unions represent the employees in negotiations and make sure their rights are upheld. Another reason why they are essential to our problem is that they have far more power than the individual employees. These are the reasons why we in this section have written a thorough review of the Danish unions in general, and more specifically, how the industrial union affect the employees at \siemens.

In Denmark, a large portion of the different rules and working conditions are made in a union agreement. A union agreement is an agreement between a trade union or a collective unit that does the bargaining with the employer, and the employer. An employer can be a single person, a company, or an employers organization.

The union agreement can include various different aspects, such as work schedule, payment, and overtime. We will now go into some of these specific agreements under the industrial collective agreement, that Siemens is a part of. It should also be noted that although a company is part of a collective agreement, they often also have a local agreement, that specifies aspects of the collective agreement. \parencite{noauthor_collective_nodate}

\subsection{Work Hours}
The working time of a standard week in Denmark is 37 hours per week. Furthermore, the hours should be placed between 6:00 and 18:00.
It is not necessary to have 37 hours every week. It is allowed to spread the hours out as long as the average working time per week is 37 hours over a twelve-month interval. 
If in these twelve months the average working hours exceed 37 hours per week, the exceeding hours should be paid according to the rules regarding overtime. The average working time for a week over the twelve-month period may not exceed 48 hours, due to the EU work-time-directive. The employee also has the right to one break each shift that according to the local agreement of \siemens is 30 minutes if the shift exceeds five work hours. \parencite{industriens_overenskomst}

The above-mentioned section is regarding full-time employees, where no other agreements have been made. However, it is also possible to have employees work on shifting teams (skiftehold).  \parencite{industriens_overenskomst}

\subsection{Shifting teams}
Employees working on shifting teams have a different agreement than the standard full-time employees. Unlike the standard employees their work hours can be placed at all hours of the day, on any day of the week. Therefore, shifting teams also need to have a work schedule that shows the different shifts. How the plan is determined is agreed upon locally.

For a shifting team, the day is separated into 3 parts. The first part is from 6:00, unless something else is agreed upon, until 14:00. The second part is from 14:00 to 22:00 and the third part is from 22:00 to 6:00. If the shifting team always work in the same part, then the following applies:
Workers in the first part will have an average work week of 37 hours, like standard employees.
Workers in the second or third will have an average work week of 34 hours a week.

If they do not work in the same part on every shift, the average work hours need to be calculated differently. If they work in both first, second, and third shift they should work an average of 35 hours per week. If they work on first and second shift, or on first and third shift, they should have an average of 35.5 hours of work per week. And lastly, if they work on second and third shift, they should have an average of 34 hours of work per week.
It should also be noted that workers who work on second or third shift are entitled to extra payment per hour, so the overall payment should be the same.

\subsection{Overtime}
Even though it is agreed upon that overtime should be avoided as much as possible, it is inescapable in some cases. When employees do work more than is written in the contract (this can be averaged over the entire year), the additional hours have to be counterbalanced which can be done in a couple of ways. The rules will differ between different union agreements, but generally workers can either be paid for the additional hours worked, or they can take some time off to counterbalance the overtime. There is not any apparent maximum amount of unscheduled overtime.

In workplaces with varying production needs employers can use something called "systematic overtime", systematic overtime is scheduled overtime with some specific rules applied to them. The scheduled overtime cannot exceed five work hours per week and maximally be one hour per day. This must be warned at least four calendar days before the week that the systematic overtime is supposed to be done. \parencite{industriens_overenskomst}

\subsection{Local Agreement}
Local agreements are additional agreements made by the individual organizations or by smaller groups of organizations and their employees. These can address specific parts of the union agreement or be completely new agreements. For \siemens, the local agreement specifies different parts of the union agreement, such as work hours, payment, and vacation. 
In the local agreement it is specified that the work schedule for the shift teams will be made for the current work year, and will be made on the 30th of April at latest, and it needs to address the work year that begins at the first of September. The workers have the right to have the work schedule ready at latest 30th of April as to be able to plan their private lives in advance.
Furthermore, it specifies that \siemens has a weekend shift team, that works on average 24 hours a week, with twelve hours on Saturday and twelve hours on Sunday.
The local agreement also specifies the placement of vacation for the workers at \siemens. An employee gets three connected weeks of vacation between the first of May and the 30th of September, one week of vacation at Christmas, and one week that can be placed after the employees wishes if possible, according to the needs of \siemens. Furthermore, on holidays that fall outside the weekend, employees should have their work hours reduced that week by 20\% (appendix \ref{appendix:local-agreements}).
